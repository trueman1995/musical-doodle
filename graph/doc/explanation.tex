\documentclass[11pt,a4paper]{article}
\usepackage[latin1]{inputenc}
\usepackage{amsmath}
\usepackage{amsfonts}
\usepackage{amssymb}
\usepackage{graphicx}

\author{Felix Armbruster}
\title{Erkl�rung zu CycleSearch}

\begin{document}
	\maketitle
	Meine L�sung der Aufgabe geht folgenderma�en vor: Zuallererst werden alle ben�tigten variablen zur�ckgesetzt, dann wird die rekursive Suche nach einem Zyklus gestartet. Startend beim ersten Knoten (dieser wird als besucht markiert) in der Knotenliste des Graphen werden alle von diesem ersten Knoten erreichbaren Knoten betrachtet. Falls die globale Variable stopRecursion nicht gesetzt ist, wird weiter vorgegangen. F�r jeden dieser Knoten wird nun �berpr�ft, ob er bereits besucht wurde. Falls nicht, ist dies der Ausgangsknoten f�r eine weitere Suche, falls ja, wird stopRecursion auf wahr gesetzt und letztendlich wahr als Ergebnis zur�ckgegeben. Falls dieser letzte Fall nie eintritt hat der Graph keinen Zyklus und es wird letztendlich falsch zur�ckgeliefert.
\end{document}