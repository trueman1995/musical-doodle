\documentclass[11pt,a4paper]{article}
\usepackage[latin1]{inputenc}
\usepackage{amsmath}
\usepackage{amsfonts}
\usepackage{amssymb}
\usepackage{graphicx}

\author{Felix Armbruster}
\title{Erkl�rung zu CycleSearch}

\begin{document}
	\maketitle
	Meine L�sung der Aufgabe geht folgenderma�en vor: Zuallererst werden alle ben�tigten variablen zur�ckgesetzt, dann wird die rekursive Suche nach einem Zyklus gestartet. Startend beim ersten Knoten (dieser wird als in Arbeit markiert, falls dieser bereits als in Arbeit markiert wurde, haben wir einen Zyklus, da wir w�hrend dieses Arbeitsprozesses wieder beim Ausgangsknoten gelandet sind, in diesem Fsll wird wahr zur�ckgeliefert) in der Knotenliste des Graphen werden alle von diesem ersten Knoten erreichbaren Knoten betrachtet. Falls die globale Variable hasCycles nicht gesetzt ist, wird weiter vorgegangen. Jeder ist der Ausgangsknoten f�r eine weitere Suche. Falls nie zu einem noch in arbeit befindlichen Knoten zur�ckgekehrt wird, hat der Graph keinen Zyklus und es wird letztendlich falsch zur�ckgeliefert.
\end{document}